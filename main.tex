\documentclass[12pt]{beamer}

\input{lecture_preamble}

\title[]{New Computational Methods\\ for Drug Design}
% \subtitle[]{Fast quantum mechanics for bio-chemistry}
\institute[, University of Copenhagen]{Department of Chemistry \\ University of Copenhagen}
\author[J. C. Kromann]{Jimmy Charnley Kromann}
\date{
    \href{http://jimmy.charnley.dk}{\tt \scriptsize jimmy.charnley.dk}
    \\[-4pt]
    \href{http://github.com/charnley}{\tt \scriptsize github.com/charnley}
}


\begin{document}

{
\usebackgroundtemplate{}
\frame[plain]{\titlepage}
}

%%%%%%%%%%%%%%%%%%%%%%%%%%%%%%%%%%%%%%%%%%%%%%%%%%%%%%%%%%%%%%%%%%%%%%
%% BEGIN FRAMES
%%%%%%%%%%%%%%%%%%%%%%%%%%%%%%%%%%%%%%%%%%%%%%%%%%%%%%%%%%%%%%%%%%%%%%

\begin{frame}

    \frametitle{What is the problem?}

    Proteins are sooo big

\end{frame}


\begin{frame}

    \frametitle{Available tools}

    current methods, much slow

\end{frame}


\begin{frame}

    \frametitle{What is the solution?}

    MM, SQM, FMO

\end{frame}


\frame
{
    \frametitle{Good Artists Copy; Great Artists Steal}

    \centering

    \begin{align*}
        { \overbrace{E(\mathrm{PM6\text{-}D3H+})}^\text{Kromann et al} } =
        { \overbrace{ E(\mathrm{PM6}) }^\text{Stewart} } +
        { \overbrace{ E(\mathrm{D3}) }^\text{Grimme et al} } +
        { \overbrace{ E(\mathrm{H+}) }^\text{Korth} }
    \end{align*}

    % $$
    % E(\mathrm{PM6\text{-}D3H+}) =
    % E(\mathrm{PM6}) +
    % E(\mathrm{D3}) +
    % E(\mathrm{H+})
    % $$

    \bigskip

    % {\scriptsize
    %     Kromann {\em et al} ({\bf 2014}) PeerJ 2:e449 http://dx.doi.org/10.7717/peerj.449
    % }

    {\scriptsize
        {\bf J. C. Kromann}, A. S. Christensen, C. Steinmann, M. Korth and J. H. Jensen\\
        PeerJ
        {\bf 2014}
        2:e449
        {10.7717/peerj.449}
    }

    % title = {{A third-generation dispersion and third-generation hydrogen bondingcorrected PM6 method: PM6-D3H+}},

    \bigskip

    \bigskip

    \bigskip

    HF-3c/FMO Interface

    \bigskip

    {\scriptsize
        {\bf J. C. Kromann} and J. H. Jensen\\
        Unpublished
    }

}


\frame
{
    \frametitle{Yet another fitted method}

    \begin{columns}[c]
        \column{0.7\linewidth}

            \centering

            {\small Interaction energies [kcal/mol]}

            \bigskip
            {

            \footnotesize

            \begin{tabular}{ @{} l r r r r @{} }
                \ra{1.3}
                & {\color{red}PM6}
                & {\color{blue}PM6-DH+}
                & {\color{kugreen}PM6-D3H+}
                & HF-3c \\
                \midrule
                \multicolumn{5}{c}{\small All Complexes} \\
                \midrule

                RMSD & {\color{red}3.35} & {\color{blue}0.80} & {\color{kugreen}0.83} & {0.54} \\
                 MAD & {\color{red}2.85} & {\color{blue}0.61} & {\color{kugreen}0.60} & {0.39} \\

                && \\
                \multicolumn{5}{c}{ \small Dispersion Complexes} \\
                \midrule

                RMSD & {\color{red}3.15} & {\color{blue}0.49} & {\color{kugreen}0.48} & {0.58} \\
                 MAD & {\color{red}2.79} & {\color{blue}0.42} & {\color{kugreen}0.36} & {0.43} \\

                && \\
                \multicolumn{5}{c}{ \small Hydrogen-bond Complexes} \\
                \midrule

                RMSD & {\color{red}4.29} & {\color{blue}0.98} & {\color{kugreen}1.11} & {0.63} \\
                 MAD & {\color{red}3.65} & {\color{blue}0.80} & {\color{kugreen}0.92} & {0.51} \\

            \end{tabular}

            }


        \column{0.35\linewidth}

            {

            \includegraphics[width=1.0\linewidth]{figures/s22_13.png}

            }


    \end{columns}

}



\begin{frame}
    \frametitle{Structure refinement of small proteins}
    
    \begin{center}
    PM6-D3H+
    \end{center}

    \begin{columns}[t]
        \column{0.5\linewidth}

            {\bf 1L2Y } 304 atoms\\
            RMSD 0.83 \AA

            \bigskip

            \includegraphics[width=1.0\linewidth]{figures/1L2Y_pm6-d3h+}

        \column{0.4\linewidth}


            {\bf 1UAO } 138 atoms\\
            RMSD: 0.93 \AA

            \bigskip

            \includegraphics[width=0.8\linewidth]{figures/1UAO_pm6-d3h+}



    \end{columns}

\end{frame}



\begin{frame}

    \frametitle{Refining Proteins Structures}

    % \begin{tabular}{@{} l l r r r r r r r r r @{} }
    %     \multicolumn{2}{c}{} &
    %     \multicolumn{4}{c}{PM6-DH+} &
    %     \multicolumn{4}{c}{PM6-D3H+}\\
    %     \cmidrule(l){3-6} \cmidrule(l){7-10}
    %     
    %     System & PDB & RMSD [\AA ] & Time [h] & Steps & $N_i$ & RMSD [\AA ] & Time [h] & Steps & $N_i$ \\
    %     
    %     \midrule
    %     
    %     Chignolin & 1UAO & 0.90 & 0.1 & 739 & 4 & 0.98 & 0.2 & 204 & 0 (0) \\
    %     Trp-Cage  & 1L2Y & 1.89 & 1.1 & 1774 & 2 & 1.61 & 5.4 & 481 & 2 (0)\\


    %     % SOLVENT

    %     \midrule
    %     
    %     Chignolin & 1UAO & 1.14 & 0.1 & 941 & 5 & 0.56 & 0.6 & 128 & 3 (0) \\
    %     Trp-Cage  & 1L2Y & 1.23 & 0.6 & 882 & 12 & 0.83 & 5.2 & 174 & 2 (0) \\
    %     
    %     \midrule
    %     
    % \end{tabular}


    % Number of imaginary frequencies for OPTTOL = 5 × 10−4 (1 × 10−4 ) aus.

    \begin{columns}

        \column{0.7\linewidth}

        \centering

        \begin{tabular}{@{} l r r r r r @{} }

            PDB & RMSD [\AA] & Time [h] & Steps & $N_i$\\

            \midrule

            &&\\

            \multicolumn{5}{c}{ \small PM6-DH+} \\

            \midrule

            % 1UAO & 0.90 & 0.1 & 739 & 4 \\
            % 1L2Y & 1.89 & 1.1 & 1774 & 2 \\

            1UAO & 1.14 & 0.1 & 941 & 5 \\
            1L2Y & 1.23 & 0.6 & 882 & 12 \\

            &&\\

            \multicolumn{5}{c}{ \small PM6-D3H+} \\

            \midrule

            1UAO & 0.56 & 0.6 & 128 & 0 \\
            1L2Y & 0.83 & 5.2 & 174 & 0 \\
            
            &&\\

            \multicolumn{5}{c}{ \small HF-3c/FMO} \\

            \midrule

            1UAO & 0.83 & 27.9$^b$ & 186 &  n/a \\


        \end{tabular}


        \column{0.4\linewidth}

            \includegraphics[width=0.8\textwidth]{figures/1L2Y.png} \\

            \bigskip

            \bigskip

            {
            \scriptsize

            1UAO: 138 atoms\newline
            1L2Y: 304 atoms\\

            \bigskip

            Ref: FMO2-RHF-D3/\newline
            6-31G(d)/PCM 

            }

            \bigskip

            {\scriptsize
                $b$ 8 cores
            }


    \end{columns}

\end{frame}






\begin{frame}
    \frametitle{Further development}

    
    \begin{columns}[t]
        \column{0.5\linewidth}

            {\bf Work in progress: }

            \begin{itemize}

                \item Accurate ligand-host docking

                \item Enzyme TS prediction

            \end{itemize}

        \column{0.4\linewidth}

            {\bf Next 3 years: }

            \begin{itemize}

                \item $d$-integrals $\in$ GAMESS

                \item PM6-3c
                \item PM6/FMO

            \end{itemize}

    \end{columns}

\end{frame}


\begin{frame}
    \frametitle{Shameless self-promotion}

    \begin{columns}[t]
        \column{0.45\linewidth}


            \begin{itemize}

                \item Molecule Calculator (molcalc.org)

            \end{itemize}

            {\scriptsize
                J. H. Jensen and {\bf J. C. Kromann}\\
                J. Chem. Edu.
                {\bf 2013}
                90(8), 1093-1095
                10.1021/ed400164n

                % (2013). The Molecule Calculator: A Web Application for Fast Quantum Mechanics-Based Estimation of Molecular Properties. Journal of Chemical Education, 90(8), 1093-1095.
            }


        \column{0.5\linewidth}

            \begin{itemize}

                \item github.com/charnley

                \begin{itemize}

                    \item Kabsch RMSD algorithm

                \end{itemize}

                \item github.com/jensengroup

                \begin{itemize}

                    \item molcalc

                    \item H+

                    \item Optimized Protein Structures

                \end{itemize}


            \end{itemize}


    \end{columns}

\end{frame}



\begin{frame}
    \frametitle{Thank you!}

    \begin{columns}[c]

        \column{0.6\linewidth}


            \begin{itemize}

                \item Jan "Yoda" Jensen (supervisor)

                \item Martin Korth (collaborator)

                \item Anders Christensen, Casper Steinmann and Lars Bratholm
                    (extreme caffeine boys)

                \item Kurt Mikkelsen, Stephan Sauer and Sten Rettrup
                    (local prof.)

            \end{itemize}

        \column{0.4\linewidth}

            \includegraphics[width=1.0\linewidth]{figures/larsandanders.jpg}

    \end{columns}


\end{frame}




%%%%%%%%%%%%%%%%%%%%%%%%%%%%%%%%%%%%%%%%%%%%%%%%%%%%%%%%%%%%%%%%%%%%%%
%% END FRAMES
%%%%%%%%%%%%%%%%%%%%%%%%%%%%%%%%%%%%%%%%%%%%%%%%%%%%%%%%%%%%%%%%%%%%%%

\end{document}

